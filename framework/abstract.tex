\chapter*{Abstract\markboth{Abstract}}
\addcontentsline{toc}{chapter}{Abstract}

	The proposed SHiP experiment aims to search for beyond standard model particles at the intensity frontier. To provide high efficiency background tagging capabilities, a \SI{50}{\meter} background tagger, consisting of $\order{2000}$ detector cells is planned to surround the decay volume of the \ac{SHiP} experiment to identify background events.
	The baseline-technology for this surround background tagger uses a liquid scintillator with two wavelength-shifting optical modules per cell, each coupled to a ring array of silicon photomultipliers. 
	To increase the spatial resolution of such a cell detector, this thesis studies the silicon photomultipliers' light yield response as a function of the particle hit position, using cosmic muons triggered by plastic scintillators.
	
	In this thesis, a cell-like prototype is used to analyse the response of the detector, operated by a wavelength shifting optical module and silicon photomultipliers, to several cosmic muon-induced event locations. 
	It was found that for a given hit location, the position of this location relative to the silicon photomultipliers can be resolved to about \SI{90}{\degree} using the mean photon impact angle $\overline{\Phi_{ew}}$. However, to make any inferences to the location of a single event from the detector's response, the optical coupling between the wavelength-shifting optical module and the silicon photomultipliers will have to be studied further.