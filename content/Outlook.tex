\chapter{Summary and Outlook}\markboth{Summary and Outlook}{}

	In this thesis, a plastic scintillator setup, located at \ac{HU} Berlin, was used to trigger cosmic muons to study the position-dependent detector response of a liquid scintillator detector instrumented with a \ac{WOM} and a ring array of \acsp{SiPM}. This detector is a prototype for the surround background tagger of the \ac{SHiP} experiment proposal to search for light feebly interacting particles at a beam-dump facility at \ac{CERN}'s \ac{SPS}. 	
	%The \ac{SHiP} experiment was proposed to study \ac{BSM} problems at the intensity frontier, making it necessary to utilise effective background suppression. 
	
	The cosmics setup includes a box filled with liquid scintillator, in which a \ac{WOM} tube transports light, which had been previously generated by cosmic muons and then wavelength-shifted by a \ac{WLS} paint, towards a photodetector array of 40 \acsp{SiPM}. The \ac{SiPM} data is collected and evaluated for different muon incidence positions, which are determined using an additional set of two plastic scintillators located on top and underneath the detector box. 
	
	The analysis of the measurement runs taken during this thesis revealed, that while for a high number of events in one position, the angle of this position relative to the \ac{SiPM} array can be resolved to about \SI{90}{\degree} using the mean photon angle $\overline{\Phi_{ew}}$, no such statements can be made with certainty about individual events. Additionally, it was shown that the optical coupling between \ac{WOM} tube and detector array was biased toward a group of \ac{SiPM} on the array. This optical coupling also influences the distribution of $\overline{\Phi_{ew}}$ values for different positions, which means it has to be studied in detail to improve the understanding of the detector's position-dependent response.
	
	This could be done by considering only events located in the centre of the \ac{WOM} tube - light generated here should be uniformly detected by all photodetectors, and any variations from that would stem from optical coupling or deformation effects. This would most likely be difficult using cosmic muons, as the determination of their incidence location is for now not very precise and the number of muons crossing a certain location makes for very long measurement times, but this problem could be circumvented by placing e.g. a $\beta$-source at the desired location. 
	
	Here the next step would also be to make the analysis of the C-0 position independent of the total number of detected photons. For this, the formulas leading to \ref{eq:phiew} would have to be adjusted to use the normalised light yield $\overline{J_i}$. Additionally, the factor $\frac{1}{N}$, where $N$ is the number of events, will have to be re-introduced to the arithmetic means, where it had been previously left out since it would cancel in the following quotient. With
	\begin{align*}
		\overline{J_i} = \frac{J_i}{\sum_{k=0}^{7} J_k},
	\end{align*}
	
	we then calculate $\overline{x_i}$ and $\overline{y_i}$ for each \ac{SiPM} channel $i$ for the C-0 position:
	
	\begin{align*}
		\overline{x_i} = \sum_{n=1}^{N} \cos(\varphi_{i,n}) \cdot \frac{J_{i,n}}{\sum_{k=0}^{7} J_{i,k}} \cdot  \frac{1}{N} \\
		\overline{y_i} = \sum_{n=1}^{N} \sin(\varphi_{i,n}) \cdot \frac{J_{i,n}}{\sum_{k=0}^{7} J_{i,k}} \cdot \frac{1}{N} \\
	\end{align*}

	These averages, now independent of the total number of photons detected and therefore independent of a specific event, can then be used to correct the $x_i$ and $y_i$ for each event in the other positions.
	
	Disregarding the optical coupling, the analysis of the position-dependent response of the detector was also influenced by the precision to which events could be located using the plastic scintillators. %\todo[inline]{this sentence position width eh} 
	These positions were not precise points on the scintillators, but had to allow for a certain width in muon locations to account for a reasonable number of events in the analysis, which meant that the $\overline{\Phi_{ew}}$ values measured in the end also correspond to relatively large ranges in the angle relative to the \ac{SiPM} array. This could be avoided by either increasing the measurement time spans to collect more events, to then be able to narrow down the positional intervals, or by repeating the measurement using a source of scintillation light that is easier to control.
	
%	\todo[inline]{unabhängig machen von light yield durch mean light yield siehe korrekturen}
	
